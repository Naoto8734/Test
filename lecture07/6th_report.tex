\documentclass[a4paper]{jsarticle}

\usepackage{otf}	%日本語化
\usepackage[dvipdfmx]{hyperref}	%ハイパーリンク
\usepackage{pxjahyper}	%ハイパーリンク
\usepackage{bm}	%数式太字
\usepackage{listings}	%プログラム埋め込み

\lstset{
	language={C}, %プログラム言語
	basicstyle={\ttfamily\small}, %書体の指定
	frame=tRBl, %フレームの指定
	framesep=10pt, %フレームと中身(コード)の間隔
	breaklines=true, %行が長くなった場合の改行
	linewidth=12cm, %フレームの横幅
	lineskip=-0.5ex, %行間の調整
	tabsize=2, %Tabを何文字幅にするかの指定
	numbers=left, %行番号を表示
	numbersep=20pt
}

\title{第6回 プログラミング応用レポート}
\author{15302114番 山下尚人}
\date{提出日:2017年月日}

\begin{document}
\maketitle%タイトル

\section{ファイル入出力(ex101.c)}
	\begin{itemize}
	\item ソースコード
		\lstinputlisting{ex101.c} 
		\mbox{}\newline
	\item 入力データファイル(exdata.dat)
		\lstinputlisting{exdata.dat} 
		\mbox{}\newline
	\item 実行結果
		\begin{lstlisting}
Input file name >>>exdata.dat
SUM=2007 AVE=167.25
		\end{lstlisting}
		\mbox{}\newline
	\item 考察\mbox{}\\
		 11~16行目で標準入力から入力された文字列のファイルを、FILE型変数fpに読み込んでいる。
		入力された文字列は、8行目で宣言されたchar型変数FileNameに、12行目のscanfにより読み込んでいる。
		読み込みに失敗した場合、15行目でプログラムを異常終了している。
		 18~24行目でfpから1行ずつデータを読み込み、int型整数sumに加算している。
		 0の行が現れたら、21行目のbreakによりforループを終了している。
		  26~28行目では、データを合計した結果と、平均を標準出力に出力している。
	\end{itemize}
	\newpage	%改ページ

\section{ファイル入出力(ex102.c)}
	\begin{itemize}
	\item ソースコード
		\lstinputlisting{ex102.c} 
		\mbox{}\newline
	\item 入力データファイル(exdata1.dat)
		\lstinputlisting{exdata1.dat} 
		\mbox{}\newline
		exdata.dataの最終行の0を削除してある
	\item 実行結果
		\begin{lstlisting}
Input file name >>>exdata1.dat
SUM=2007 AVE=167.25
		\end{lstlisting}
		\mbox{}\newline
	\item 考察\mbox{}\\
		11~16行目で標準入力から入力された文字列のファイルを、FILE型変数fpに読み込んでいる。
		入力された文字列は、8行目で宣言されたchar型変数FileNameに、12行目のscanfにより読み込んでいる。
		読み込みに失敗した場合、15行目でプログラムを異常終了している。
	\end{itemize}
	\newpage	%改ページ


\end{document}
