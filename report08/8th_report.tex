\documentclass[a4paper]{jsarticle}

\usepackage{otf}	%日本語化
\usepackage[dvipdfmx]{hyperref}	%ハイパーリンク
\usepackage{pxjahyper}	%ハイパーリンク
\usepackage{bm}	%数式太字
\usepackage{listings}	%プログラム埋め込み

\lstset{
	language={C}, %プログラム言語
	basicstyle={\ttfamily\small}, %書体の指定
	frame=tRBl, %フレームの指定
	framesep=10pt, %フレームと中身(コード)の間隔
	breaklines=true, %行が長くなった場合の改行
	linewidth=12cm, %フレームの横幅
	lineskip=-0.5ex, %行間の調整
	tabsize=2, %Tabを何文字幅にするかの指定
	numbers=left, %行番号を表示
	numbersep=20pt
}

\title{第6回 プログラミング応用レポート}
\author{15302114番 山下尚人}
\date{提出日:2017年12月15日}

\begin{document}
\maketitle%タイトル

\section{リスト}
	\begin{itemize}
	\item ソースコード
		\lstinputlisting{list1.c} 
		\mbox{}\newline
	\item 実行結果
		\begin{lstlisting}
value=100
value=20
value=10
value=5
value=2
value=1
		\end{lstlisting}
		\mbox{}\newline
	\item 考察\mbox{}\\
		 考察
	\end{itemize}
	\newpage	%改ページ

\section{リスト}
	\begin{itemize}
	\item ソースコード
		\lstinputlisting{list2.c} 
		\mbox{}\newline
	\item 実行結果
		\begin{lstlisting}
value=1
value=2
value=5
value=10
value=20
value=100
		\end{lstlisting}
		\mbox{}\newline
	\item 考察\mbox{}\\
		こうさつ
	\end{itemize}
	\newpage	%改ページ


\end{document}
