\documentclass[a4paper]{jsarticle}

\usepackage{otf}	%日本語化
%\usepackage[dvipdfmx]{hyperref}	%ハイパーリンク
%\usepackage{pxjahyper}	%ハイパーリンク
\usepackage{bm}	%数式太字
\usepackage{listings}	%プログラム埋め込み

\lstset{
	language={C}, %プログラム言語
	basicstyle={\ttfamily\small}, %書体の指定
	frame=tRBl, %フレームの指定
	framesep=10pt, %フレームと中身(コード)の間隔
	breaklines=true, %行が長くなった場合の改行
	linewidth=12cm, %フレームの横幅
	lineskip=-0.5ex, %行間の調整
	tabsize=2, %Tabを何文字幅にするかの指定
	numbers=left, %行番号を表示
	numbersep=20pt
}

\title{第回 プログラミング応用レポート}
\author{15302114番 山下尚人}
\date{提出日:2017年12月8日}

\begin{document}
\maketitle%タイトル

\section{構造体の配列}
	\begin{itemize}
	\item ソースコード
		\lstinputlisting{ex84.c} 
		\mbox{}\newline
	\item 実行結果
		\begin{lstlisting}
Please input No, x, y.
1 1 2
2 0 0
3 9 9
4 5 5
5 7 2
最も離れた座席は2と3です。(距離:12.727922)
		\end{lstlisting}
		\mbox{}\newline
	\item 考察\mbox{}\\
		7~11行目で構造体のseat\_position型を宣言し、MAX\_SEAT個の配列seatを宣言している。\\
		21~23行目で構造体の配列seatに標準入力からの値を、配列の各要素にアドレス演算子を用いて記録している。\\
		25~36行目で2重のfor文を用いて、seatの各要素を総当たりで2つずつ呼び出して、距離disを計算している。\\
		計算した距離disは30行目で前回までの最大値max\_disと比較し、最大値を更新している。\\
		32,33行目で最大値となった座標の番号を配列max\_dis\_seatに記録している。\\
		38行目で、2重forループを抜けた後の距離の最大値と、その番号を結果として表示している。\\
	\end{itemize}
	\newpage	%改ページ

\section{アロー演算子,構造体を引数として渡す関数}
	\begin{itemize}
	\item ソースコード
		\lstinputlisting{ex91.c} 
		\mbox{}\newline
	\item 実行結果
		\begin{lstlisting}
x = 10, y=2.000000e+01
		\end{lstlisting}
		\mbox{}\newline
	\item 考察\mbox{}\\
		4~7行目で構造体のtag型を宣言し、tag型の変数sampleを宣言している。\\
		9行目でtag型の変数pを間接演算子をつけて宣言している。\\
		11行目では、アドレス演算子を用いて変数sampleのアドレスを、pに代入している。\\
		12,13行目では、アロー演算子を用いて、tag型構造体pの内部の変数x,yのアドレスにそれぞれ数値を代入している。
		ここで、pはsampleのアドレスが代入されているので、sampleのアドレスと同義となる。
	\end{itemize}
	\newpage	%改ページ

\end{document}
