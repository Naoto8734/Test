\documentclass[a4paper]{jsarticle}

\usepackage{otf}	%日本語化
\usepackage[dvipdfmx]{hyperref}	%ハイパーリンク
\usepackage{pxjahyper}	%ハイパーリンク
\usepackage{bm}	%数式太字
\usepackage{listings}	%プログラム埋め込み

\lstset{
	language={C}, %プログラム言語
	basicstyle={\ttfamily\small}, %書体の指定
	frame=tRBl, %フレームの指定
	framesep=10pt, %フレームと中身(コード)の間隔
	breaklines=true, %行が長くなった場合の改行
	linewidth=12cm, %フレームの横幅
	lineskip=-0.5ex, %行間の調整
	tabsize=2, %Tabを何文字幅にするかの指定
	numbers=left, %行番号を表示
	numbersep=20pt
}

\title{第7回 プログラミング応用レポート}
\author{15302114番 山下尚人}
\date{提出日:2017年12月22日}

\begin{document}
\maketitle%タイトル

\section{ex103.c}
	\begin{itemize}
	\item ソースコード
		\lstinputlisting{ex103.c} 
		\mbox{}\newline
	\item 入力データファイル(exdata1.dat)
		\lstinputlisting{exdata1.dat} 
		\mbox{}\newline
	\item 実行結果
		\lstinputlisting{logdata1.log} 
		\mbox{}\newline
	\item 出力データファイル(out.dat)
		\lstinputlisting{out.dat} 
		\mbox{}\newline
	\item 考察\mbox{}\\
		10〜13行目でexdata1.datを"r"モードでfpinに展開している。\\
		15〜19行目でout.datを"w"モードでfpoutに展開している。\\
		21〜25行目でfpinから1行ずつデータをvalに読み込み、valをfpoutに書き込んだ後、sumにvalを加算している。\\
		27〜29行目でfpinのデータの合計と平均を標準出力に出力している。
	\end{itemize}
	%\newpage	%改ページ

\section{ex104:append(追記)の場合}
	\begin{itemize}
	\item ソースコード
		\lstinputlisting{ex104_1.c} 
		\mbox{}\newline
	\item 実行結果
		\lstinputlisting{logdata2.log} 
		\mbox{}\newline
	\item コードの考察\mbox{}\\
		7〜10行目でsample.datを"a"モードでfpに展開している。\\
		13,14行目でfpに文字列を書き込んでいる。
	\item 実行結果の考察\mbox{}\\
		1〜10行目でsample.datがない状態でコードを実行すると、文字列が1回書き込まれたsample.datが作られる。\\
		11〜20行目でsample.datがある状態でコードを実行すると、文字列が末尾に新たに書き込まれる。
	\end{itemize}
	%\newpage	%改ページ

\section{ex104:write(上書き)の場合}
	\begin{itemize}
	\item ソースコード
		\lstinputlisting{ex104_2.c} 
		\mbox{}\newline
	\item 実行結果
		\lstinputlisting{logdata3.log} 
		\mbox{}\newline
	\item コードの考察\mbox{}\\
		2の追記の場合のコードから、7行目の"a"を"r"に変更した。
	\item 実行結果の考察\mbox{}\\
		1〜16行目でsample.datがある状態でコードを実行すると、元のデータが消え、新たに文字列が1回書き込まれる。\\
		17〜20行目でsample.datを消している。\\
		21〜28行目でsample.datがない状態でコードを実行すると、sample.datが新たに作成され、文字列1回が書き込まれる。
	\end{itemize}
	\newpage	%改ページ
	

\end{document}
