\documentclass[a4paper]{jsarticle}

\usepackage{otf}	%日本語化
\usepackage[dvipdfmx]{hyperref}	%ハイパーリンク
\usepackage{pxjahyper}	%ハイパーリンク
\usepackage{bm}	%数式太字
\usepackage{listings}	%プログラム埋め込み

\lstset{
	language={C}, %プログラム言語
	basicstyle={\ttfamily\small}, %書体の指定
	frame=tRBl, %フレームの指定
	framesep=10pt, %フレームと中身(コード)の間隔
	breaklines=true, %行が長くなった場合の改行
	linewidth=12cm, %フレームの横幅
	lineskip=-0.5ex, %行間の調整
	tabsize=2, %Tabを何文字幅にするかの指定
	numbers=left, %行番号を表示
	numbersep=20pt
}

\title{第10回 プログラミング応用レポート}
\author{15302114番 山下尚人}
\date{提出日:2018年1月9日}

\begin{document}
\maketitle%タイトル

\section*{課題}
	\begin{itemize}
	\item ソースコード
		\lstinputlisting{list6.c} 
		\mbox{}\newline
	\item data1.dat
		\lstinputlisting{data1.dat} 
		\mbox{}\newline
	\item data2.dat
		\lstinputlisting{data2.dat} 
		\mbox{}\newline
	\item 実行結果
		\begin{lstlisting}
value=10
value=9
value=8
value=7
value=6
value=5
value=4
value=3
value=2
value=1
		\end{lstlisting}
		\mbox{}\newpage
	\item 考察\mbox{}\\
		set\_cell関数は54〜70行目に定義されている関数。リストの先頭のセルと値を引数に渡すと、連結リストの最後に新しいセルが追加される。\\
		
		print\_cell関数は72〜78行目に定義されている関数。リストの先頭のセルを渡すと、標準出力にリストの値が順番に出力される。\\
		
		list\_concatenate関数は80〜93行目に定義されている関数。引数としてリストの先頭のセルを2つ渡すと(data0,data1)、data0のリストの最後にdata1のリストが連結する。\\
		data1のリストがある時(83行目のif文)はdata0を連結し、無い時は何もしない。\\
		data0のリストがある時(84行目のif文)は、85〜87行目のfor文でdata0のリストの最後のセルを探し、最後セルのnextをセルdata1のnextに書き換えて更新している。\\
		data0のリストが無い時はセルdata0のnextをセルdata1のnextに書き換えて更新している。\\
		
		13〜51行目はmain関数の処理。\\
		17〜23行目ではdata1とdata2の2つのリストのダミーヘッダの作成を行っている。\\
		24〜33行目でファイルdata1.datから値を読み込み、リストdata1にセットしている。\\
		35〜44行目でファイルdata2.datから値を読み込み、リストdata2にセットしている。\\
		47行目でdata1、data2をlist\_concatenate関数に渡して、リストを連結している。
	\end{itemize}
\end{document}
