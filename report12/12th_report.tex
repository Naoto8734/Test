\documentclass[a4paper]{jsarticle}

\usepackage{otf}	%日本語化
\usepackage[dvipdfmx]{xcolor} %色
\usepackage{listings}	%プログラム埋め込み

\lstset{
	language={C}, %プログラム言語
	basicstyle = \ttfamily\scriptsize, 	%標準の書体
	keywordstyle = {\color[cmyk]{0,1,0,0}},%キーワード(int, ifなど)の書体
	commentstyle = {\color[cmyk]{1,0.4,1,0}},%コメントの書体
	frame=tRBl, %フレームの指定
	framesep=10pt, %フレームと中身(コード)の間隔
	breaklines=false, %行が長くなった場合の改行
	linewidth=15cm, %フレームの横幅
	lineskip=-0.5ex, %行間の調整
	tabsize=2, %Tabを何文字幅にするかの指定
	numbers=left, %行番号を表示
	numbersep=20pt,
 	captionpos = b, %キャプションの場所("tb"ならば上下両方に記載)
	xleftmargin=2truecm,%左側の余白
}


\title{第12回 プログラミング応用レポート}
\author{15302114番 山下尚人}
\date{提出日:20年1月30日}

\begin{document}
\maketitle%タイトル

\section*{課題}
	\begin{itemize}
	\item 関数inorderのソースコード
		\begin{lstlisting}
void inorder(struct node *p) {
	if (p==NULL)
		return;
	inorder(p->left);
	printf("節%cに立ち寄りました\n", n2c(p->label));
	inorder(p->right);
}
		\end{lstlisting}
		\mbox{}\newline
	\item 関数postorderのソースコード
		\begin{lstlisting}
void postorder(struct node *p) {
	if (p==NULL)
		return;
	postorder(p->left);
	postorder(p->right);
	printf("節%cに立ち寄りました\n", n2c(p->label));
}
		\end{lstlisting}
		\mbox{}\newpage
	\item 実行結果
		\begin{lstlisting}
           [H]
      [A]
                     [F]
                [D]
                     [E]
                          [G]
           [B]
                [C]

---preorder---
節Aに立ち寄りました
節Bに立ち寄りました
節Cに立ち寄りました
節Dに立ち寄りました
節Eに立ち寄りました
節Gに立ち寄りました
節Fに立ち寄りました
節Hに立ち寄りました

---inorder---
節Cに立ち寄りました
節Bに立ち寄りました
節Gに立ち寄りました
節Eに立ち寄りました
節Dに立ち寄りました
節Fに立ち寄りました
節Aに立ち寄りました
節Hに立ち寄りました

---postorder---
節Cに立ち寄りました
節Gに立ち寄りました
節Eに立ち寄りました
節Fに立ち寄りました
節Dに立ち寄りました
節Bに立ち寄りました
節Hに立ち寄りました
節Aに立ち寄りました
		\end{lstlisting}
		\mbox{}
		
	\end{itemize}
\end{document}
