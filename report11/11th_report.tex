\documentclass[a4paper]{jsarticle}

\usepackage{otf}	%日本語化
\usepackage[dvipdfmx]{hyperref}	%ハイパーリンク
\usepackage{pxjahyper}	%ハイパーリンク
\usepackage{bm}	%数式太字
\usepackage[dvipdfmx]{graphicx,xcolor} %jpeg/pngの挿入
\usepackage{tikz} %図形の直接描画
\usepackage{listings,jlisting}  %コードの挿入

\lstset{
	language={C}, %プログラム言語
	%basicstyle={\ttfamily\small}, %書体の指定
	basicstyle = \ttfamily\scriptsize, 	%標準の書体
	keywordstyle = {\color[cmyk]{0,1,0,0}},%キーワード(int, ifなど)の書体
	commentstyle = {\color[cmyk]{1,0.4,1,0}},%コメントの書体
	frame=tRBl, %フレームの指定
	framesep=10pt, %フレームと中身(コード)の間隔
	breaklines=false, %行が長くなった場合の改行
	linewidth=12cm, %フレームの横幅
	lineskip=-0.5ex, %行間の調整
	tabsize=2, %Tabを何文字幅にするかの指定
	numbers=left, %行番号を表示
	numbersep=20pt
}

\title{第11回 プログラミング応用レポート}
\author{15302114番 山下尚人}
\date{提出日:2018年1月25日}

\begin{document}
\maketitle%タイトル

\section*{課題}
	\begin{itemize}
	\item ソースコード
		\lstinputlisting{tree1.c} 
		\mbox{}\newline
	\item 実行結果
		\begin{lstlisting}
***** TREE:add 8 *****
  8
3
***** TREE:add 6 *****
  8
    6
3
***** TREE:add 5 *****
  8
    6
      5
3
***** TREE:add 4 *****
  8
    6
      5
        4
3
***** TREE:add 9 *****
    9
  8
    6
      5
        4
3
***** TREE:add 2 *****
    9
  8
    6
      5
        4
3
  2
***** TREE:add 7 *****
    9
  8
      7
    6
      5
        4
3
  2
***** TREE:add 1 *****
    9
  8
      7
    6
      5
        4
3
  2
    1
		\end{lstlisting}
		\mbox{}\\

	\item プリントアウト関数の動作説明\\
		画面の垂直方向(行)の順番は、ソースの92,101行目の再帰呼び出しと98行目のprintf関数により決まる。\\
		右側のノードの再帰呼び出し、ノードの画面出力、左側のノードの再帰呼び出しの順で処理している。\\
		よって、「そのノードの右の子の末端へ移動し出力。一つ親のノードへ移動して出力し、ノードの左の子へ移動。」を繰り返している。\\
		91,101行目のif文による条件分岐は再帰呼び出しを終わらせるため。\\
		
		
		画面の水平方向(列)の空白の数は、ソースの89,104行目の変数levelの増減により処理し、95〜97行目のfor文により実際に出力している。\\
		printTree関数内のローカル変数levelは、static装飾子によりprintTree関数が呼ばれるたびに初期化されず、値が保持される。\\
		printTree関数は92,101行目で再帰呼び出しされているので、levelが1増えた後に再帰呼び出しされ、呼び出しが終わるごとにlevelが1減っていく。\\
		これにより、95〜97行目でノードのレベルの数だけ空白を画面出力した後、98行目でlabelを出力して改行している。
			
	\end{itemize}
\end{document}
