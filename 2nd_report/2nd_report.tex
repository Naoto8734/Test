\documentclass[a4paper]{jsarticle}

\usepackage{otf}	%日本語化
\usepackage[dvipdfmx]{hyperref}	%ハイパーリンク
\usepackage{pxjahyper}	%ハイパーリンク
\usepackage{bm}	%数式太字
\usepackage{listings,jlisting}	%プログラム埋め込み

\lstset{
	language={C}, %プログラム言語
	basicstyle={\ttfamily\small}, %書体の指定
	frame=tRBl, %フレームの指定
	framesep=10pt, %フレームと中身(コード)の間隔
	breaklines=true, %行が長くなった場合の改行
	linewidth=12cm, %フレームの横幅
	lineskip=-0.5ex, %行間の調整
	tabsize=2, %Tabを何文字幅にするかの指定
	numbers=left, %行番号を表示
	numbersep=20pt
}

\title{第2回 プログラミング応用レポート}
\author{15302114番 山下尚人}
\date{提出日:2017年11月10日}

\begin{document}
\maketitle%タイトル

\section{ポインタと配列}
	\begin{itemize}
	\item ソースコード
		\lstinputlisting{ptest5_1.c} 
		\mbox{}\newline
	\item 実行結果
		\begin{lstlisting}
合計=55
		\end{lstlisting}
		\mbox{}\newline
	\item 考察\mbox{}\\
		10~12行目のfor文で配列の添字を順次増やして、配列の各数字の合計を求めている。
	\end{itemize}
	\newpage	%改ページ


\section{配列内のデータの合計を求めるプログラム(ポインタを使用)}
	\begin{itemize}
	\item ソースコード
		\lstinputlisting{ptest5_2.c}
		\mbox{}\newline
	\item 実行結果
		\begin{lstlisting}
合計=55
		\end{lstlisting}
		\mbox{}\newline
	\item 考察\mbox{}\\
		10行目でpに配列の最初のアドレスを代入している。
		12~14行目のfor文でアドレスをずらしつつ、配列の値を*(間接演算子)により参照し、合計している。
	\end{itemize}
	\newpage	%改ページ


\section{Cらいしいプログラム,演算子の優先度}
	\begin{itemize}
	\item ソースコード
		\lstinputlisting{ptest6.c}
		\mbox{}\newline
	\item 実行結果
		\begin{lstlisting}
合計=55
		\end{lstlisting}
		\mbox{}\newline
	\item 考察\mbox{}\\
		4行目の配列の最後に0(ターミネータ)を入れることにより、11~13行目のwhile文が配列の最後で終了する。
		12行目では先に間接演算子により参照した後、p=p+1が行われることで次の配列の要素のアドレスへと移動する。\\
		これにより、配列の最初から最後まで順次加算された、合計が出力される。
	\end{itemize}
\end{document}